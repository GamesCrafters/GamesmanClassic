\documentclass[pdftex]{article}
\usepackage{amsmath}
\usepackage{hyperref}

% macros for underlined/colored links
\newcommand{\gcref}[1]{\textcolor{cyan}{\underline{\ref{#1}}}}
\newcommand{\gcautoref}[1]{\textcolor{cyan}{\underline{\autoref{#1}}}}
\newcommand{\gchref}[2]{\textcolor{blue}{\underline{\href{#1}{#2}}}}

\hypersetup{backref, colorlinks=true, linkcolor=cyan, urlcolor=blue}

\begin{document}
\title{Documentation for WinBy}
\author{Alan Roytman and John Lo}
\date{2006-12-14}
\maketitle
\tableofcontents
\newpage

\section{Overview}

\indent \indent WinBy is a solver and gameplay feature where the perfect computer
tries to maximize the winning score or (if it is a losing position) minimize
the number of points the opponent receives.  This is in contrast to the typical
mode of play, where the perfect computer either wins as quickly as possible
or loses as slowly as possible (i.e., remoteness).  Currently, WinBy is implemented
for the standard solver along with the variable slices aware solver.  If
the module uses the goAgain solver, the WinBy feature will not work.  In addition,
the module should use generic\_hash so the solver is aware of whose turn it is
while it is propagating values up the game tree.  For all games that support
WinBy, the user may decide whether he or she wants the computer to play according
to the WinBy optimization or the remoteness value stored in the database.  Should
the player desire to know the WinBy values for all moves from a given position, it
has been added to ``ps'' output and can be viewed during gameplay.  Currently,
WinBy is implemented for mothello and miceblocks.

\section{Design Decisions}

\indent \indent WinBy values are of type WINBY and are type defined to be integers.
This is important, because the values are allowed to be negative!  In fact,
WinBy is \emph{defined from player one's perspective}.  This has a few consequences
for both the module and the solver.  For example, WinBy values for Othello are
always calculated as (\# black pieces - \# white pieces), no matter whose turn it
is or whether the value is positive or negative.

The ramifications for the solver come into play when WINBY values are being
propagated up the tree.  If it is player one's turn for a given position,
player one wants to \emph{maximize} over all possible WINBY values of that
position's children (i.e., get as close to positive infinity as possible).
On the other hand, if it is player two's turn for a certain
position, then that player wants to \emph{minimize} over all possible WINBY
values of that position's children (i.e., get as close to negative infinity
as possible).  These are two properties that follow from the definition of WinBy.
As a result, the solver must know whose turn it is before it can do any bubbling
up intelligently.  This is precisely the reason the module needs to use generic\_hash,
since the solver cannot extract whose turn it is from the hashed POSITION.

The module's WinBy function is only called on positions which are primitives.  All
other positions have their WinBy values defined recursively.  For example, if
the initial position has a WINBY of 8, then this implies that there is a path
from root to leaf where all intermediate nodes and the leaf node have WinBy values of 8.
Actually, this path might very well be the choice of moves executed during a match if
two perfect opponents are playing each other.  Here is why: if it is player one's turn
for a given POSITION and it has a WINBY of 8, then that implies there exists a child
with a WINBY of 8 and it is the maximum possible value.  Similarly, if it is player two's
turn, then that implies there exists a child with a WINBY of 8 and it is the minimum
of all its children.  Thus, for any given position and any player, \emph{the child
with the same} WINBY \emph{value as the parent is the best move that player can make}.

Currently, memdb and bpdb are the only database files that support WinBy.  For memdb,
WINBY values are used instead of the existing MEX values since there is no game that
requires both pieces of data.  This creates a limitation on how large or small a WINBY
value can be, since MEX values are only given 5 bits.  That is, a WINBY value is only
allowed to range from -16 to 15; anything else is ill defined.

\section{Function Names}

\indent \indent The following are the main functions/globals associated with WinBy.  All other
functions are specific to the database or solver implementation.

\begin{verbatim}
globals.h:

BOOLEAN gWinBy;
\end{verbatim}
This boolean variable determines whether or not the computer should play by the WinBy or
remoteness optimization.  The default value is TRUE, which implies that any game which
supports WinBy will play by the WinBy optimization.  The user can make the value FALSE
inside Gamesman's configuration menu by pressing ``w''.

\begin{verbatim}
WINBY (*gPutWinBy)(POSITION);
\end{verbatim}
This is an API function which allows the module to announce to the core that it supports
the WinBy feature.  This function pointer needs to point to whatever function the
module uses to compute the WINBY value of a primitive position.  The function takes in
a hashed POSITION, and returns the WINBY value of that POSITION, always from player
one's perspective.

\begin{verbatim}
db.h:

void WinByStore (POSITION pos, WINBY winBy);
\end{verbatim}
This is the generic WinBy DB function that writes the WINBY value for the POSITION
in the database.  Precisely how it accomplishes this task is specific to what
database is being used.  In the case of memdb, the MEX value is overwritten, and in
the case of bpdb, a WINBY slot is added.

\begin{verbatim}
WINBY WinByLoad (POSITION pos);
\end{verbatim}
This is the generic WinBy DB function that loads the WINBY value for the given
POSITION.  In the case of memdb, it just reads the same bits that MEX normally
occupies.  In the case of bpdb, it reads from the WINBY slot.  Sign extension is
taken into account for the case of negative WINBY values.

\begin{verbatim}
gameplay.c:

MOVE GetWinByMove (POSITION position, MOVELIST* genMoves);
\end{verbatim}
Given a POSITION and a MOVELIST generated from the POSITION,
this function returns a MOVE to a child with the same WINBY value
as its parent.

\section{Module Writers}
\indent \indent Module writers need to do three things in order to properly implement
the WinBy feature.  The first is to write a function that computes the
WINBY value for a given primitive POSITION \emph{from player one's perspective}.  
This function should take as input a POSITION and return a WINBY.  Second, the module needs to set the
global function pointer gPutWinBy to the address of the function that computes
WINBY values.  Third, to properly display the value in PrintPosition, the module
should call GetPrediction.  If WinBy is activated during gameplay, GetPrediction
appropriately returns information regarding how much the player can win (or lose)
by as well as how many moves need to be made (i.e., remoteness).  If the user
decides that he or she wants the computer to play by remoteness, then no WinBy
information will be returned in the prediction string.  That is it!
Here is an example of all three steps, taken straight from mothello.c:
\begin{verbatim}
InitializeGame() {
  ...
  gPutWinBy = &computeWinBy;
  ...
}
\end{verbatim}

\begin{verbatim}
WINBY computeWinBy(POSITION position) {
  int i;
  int whitetally = 0, blacktally = 0;
  char *board = getBoard(position);  
  for(i = 0; i < (OthCols * OthRows); i++) {
    if(board[i] == WHITEPIECE)
      whitetally++;
    if(board[i] == BLACKPIECE)
      blacktally++;
  }
  return blacktally - whitetally;
}
\end{verbatim}

\begin{verbatim}
void PrintPosition (POSITION position, STRING playerName, BOOLEAN usersTurn) {
  ...
  sprintf(prediction,"| %s",GetPrediction(position,playerName,usersTurn));
  ...
}
\end{verbatim}

\section{Future Goals}
\indent \indent In the future, WinBy should be made independent of generic\_hash.  Also, it would
be nice to have a larger range of WINBY values (as of now, the range is -16 to 15).
Lastly, we would like to eventually add a feature that allows the computer to minimize the
WINBY value (but still win), in the interest of giving the human player false hope.

\end{document}
